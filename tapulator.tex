% !TEX root =  paper.tex
\section{Preliminary Study}
\label{sec:tapulator}

To evaluate the potential of the gestures, we compare the Espresso method to a standard accessible numeric entry method using 
  \begin{enumerate*}[(1) ]
    \item a theoretical analysis of the methods and 
    \item an empirical comparison with five users. 
  \end{enumerate*}
We choose the Espresso method over the Cappuccinno method because it is more intuitive to learn.

\subsection{Theoretical Analysis}
In the standard method, there are two steps involved in entering a digit. First, the user has to explore the screen to locate the button where the digit resides. The seek time is hard to quantify, but it requires listening to the buttons touched until the correct one is found. Once the button is located, the user performs a selection gesture. This can be done using a split tap, hold a finger down on the target button and use another finger to tap the screen, or a one-finger double tap anywhere on the screen. At a first attempt, this method appears to be more difficult than the 2.1 taps per digit.

\subsection{Empirical Evaluation}
Our empirical evaluation consists of a study with five sighted participants. In each of the study, the participants entered 10 six-digit numbers using the standard and the Espresso methods. Participants held the smartphone beneath the desk so that they were not able to see the screen. After a brief practice session, the participants entered the numbers at an average rate of 1.99 seconds per digit ($SD = 1.25$) using the Espresso method and at an average rate of 2.77 seconds per digit ($SD = 1.24$) with the standard VoiceOver method. The error rates were far lower for the Espresso method. The participants produced Mean String Distance (MSD) only 1\% on average using Espresso method. Whereas, using the standard method, the participants produced a 14.2\% on average of the MSD. Thus, the Espresso method out-performed the standard method. Unsurprisingly, all five participants perferred the Espresso method to the standard method \cite{Ruamviboonsuk:2012}.
\par
While the results are from a preliminary study, they show that the Espresso method has potential to out-perform the standard numeric entry method. We decided to conduct a more rigorous study not only on the Espresso method, but both the Espresso and the Cappuccino methods.