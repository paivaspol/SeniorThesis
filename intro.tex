% !TEX root =  paper.tex
\section{Introduction}
\label{sec:intro}

Visually impaired people use touchscreen on smartphones with a screen reader. Screen readers, such as Apple's VoiceOver \cite{VoiceOver:2014}, provides audio feedback for the user to navigate the phone. The user uses his finger to find the target item on the screen. The screen reader gives continuous voice feedback to the user. Once the target is reached, the user performs a selection gesture. A selection gesture can be either a split-tap or a double-tap anywhere on the screen. Since the user has to pay careful attention to the voice feedback, using a screen reader in a noisy setting is difficult. Wearing headphones is not practical because blind users feel unsafe since they use the sounds around them to navigate and understand their surroundings \cite{Azenkot:2011}.
\par
We introduce \textit{DigiTaps}; an eyes-free number entry method that uses minimal voice feedback. The digits are represented by a combination of simple gestures such as tap and swipe. Users can perform the gestures to enter the digits anywhere on the screen. For example, the number 20 is entered by performing a two-finger tap and a swipe anywhere on the screen. We conducted a preliminary study to compare the DigiTaps method to a numeric keyboard with a screen reader. The participants achieved a significantly lower error rate by using DigiTaps method than using the screen reader method, while still achieving a high entry rate.
\par
In addition to the preliminary evaluation, a longitudinal lab study was conducted to compare the two DigiTaps codes when the codes are used with a voice and haptic feedback together and when the codes are used with only haptic feedback. The lab study shows that the participants can enter numbers fast and accurate using the DigiTaps number entry method and the DigiTaps codes are good alternatives to using a numeric keyboard with screen reader. \cite{Azenkot:2013}
\par
Even though a longitudinal lab study was conducted, we would like to evaluate the gestures in a real-world setting. To emulate a real-world use of the gesture, we developed DigiTaps game for conducting number entry method user studies in the wild. We distribute the DigiTaps game via Apple App Store to expose it to a large number of players. The player has to enter the numbers shown by the game using one of the DigiTaps code. The game consists of 5 levels and the difficulty increases as the player progresses through the levels. We collect data at every touch event that the player has performed along with the metadata information such as the indicator of the beginning of the level, the number presented to the player and what number was shown to the players. We use the data for evaluating the DigiTaps codes.

\begin{comment}
Outline:
  - Background on why we come up with DigiTaps code: Continuous feedback hard to use in a noisy setting.
  - evidence of other method presented
  - why DigiTaps game? DigiTaps game is for what?

\end{comment}
