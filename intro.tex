% !TEX root =  paper.tex
\section{Introduction}
\label{sec:intro}

Visually impaired people uses touchscreen on smartphone with screen reader. Screen readers, such as Apple VoiceOver \cite{VoiceOver:2014}, provides audio feedback for the user to navigate the phone. The user uses his finger to find the target item on the screen. The screen gives continuous voice feedback. Once the target is reached, the user performs a selection gesture, either a split-tap or a double-tap anywhere on the screen. We can see that using screen reader in a noisy setting is difficult, since the user has to pay careful attention to the voice feedback. Wearing headphones is not an option because blind users feel unsafe since they use the sounds around them to navigate and understand their surroundings \cite{Azenkot:2011}.
\par
We introduce DigiTaps, an eyes-free number entry method that uses minimal voice feedback. Users can enter digits by tap or swipe anywhere on the screen. The digits are represented by two different DigiTaps codes. For example, the digit 2 is entered by performing a two-finger tap anywhere on the screen. We conducted a preliminary study to compare DigiTaps method to the screen reader method. The participants achieved a significantly lower error rate by using DigiTaps method than using the screen reader method.
\par
We developed DigiTaps game for conducting number entry method in the large. DigiTaps Game is straight forward. It shows a number to the player and the player uses DigiTaps code to enter the number. We collect data at every touch event that the player has performed along with the information on whether the player entered the number correctly or not. We plan to use the data collected for further analysis on the evaluation of the DigiTaps code.

\begin{comment}
Outline:
  - Background on why we come up with DigiTaps code: Continuous feedback hard to use in a noisy setting.
  - evidence of other method presented
  - why DigiTaps game? DigiTaps game is for what?

\end{comment}