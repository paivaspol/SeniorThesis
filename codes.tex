% !TEX root =  paper.tex
\section{DigiTaps Codes}
\label{sec:codes}

Numbers are represented in different ways using different numeral representation system such as base-2 and base-10. Numbers that we use on a daily basis are mostly represented in base-10 numeric system. If we represent the first 10 digits, 0 through 9, in base-5, the digits are represented as 0, 1, 2, 3, 4, 10, 11, 12, 13, 14. We call these representations \textit{codes}. There are two types of codes fixed length and variable length. All the codes represented by fixed length have the same length. For example, if 0 through 9 are represented with fixed-length code in base-5, we write them as 00, 01, 02, 03, 04, 10, 11, 12, 13, 14 and all the codes are 2 digits long.
\par
In contrary, variable length codes can have variable length. For example, to write the first nine digits, 0 through 9, in base-5 system using variable length code, we write them as 0, 1, 2, 3, 4, 10, 11, 12, 13, 14. The first five codes are 1-digit long and the rest of the codes are 2 digits long. Variable-length codes are shorter on average. Since the variable length codes are potentially shorter than fixed length codes on average, we decided to use variable-length codes for representing the numbers.
\par
Although variable length codes can represent numbers with shorter codes than fixed length codes. It introduces an ambiguity to the codes. For example, suppose we have a program to decode base-5 variable length codes to base-10 digits. We enter the number 1 into the program. The program cannot determine whether the number 1 suppose to be the code 1 or the first digit of 11. The problem does not arise in fixed length codes because all the code have the same length. Thus, the digit 1 is represented by 01. To resolve the ambiguity, we use prefix free code, a variant of variable length code.
\par
Prefix free codes do not allow a code to be a prefix of another code \cite{huf52}. In our example above, if we use prefix free code to represent 0 through 9 in base-5, there cannot be 10, 11, 12, 13, 14 in the codes because 1 is already a code by itself and it cannot be a prefix of another code. Since using prefix free code removes the ambiguity among the codes and it is a variable length code, we use prefix free code to represent our codes in DigiTaps.

\subsection{Espresso}
We develop an easy to learn prefix free code called Espresso. With Espresso, the digits can be derived by adding the numbers 0, 1, 2, and 3. The numbers 0, 1 and 2 are represented by one finger swipe, one-finger tap, and two-finger tap, respectively. 3 is represented by three-finger tap and a swipe, $(3+0)$. Similarly, 4 is represented by three-finger tap and a one-finger tap, $(3+1)$. 6 is represented with 2 two-finger taps and one-finger swipe, $(3+3+0)$. The final swipe is required to ensure that the code is prefix-free. Finally, 9 is represented by 3 three-finger taps, $(3+3+3)$. A swipe is not necessary for 9 because 3 three-finger taps is not a prefix to another input (see table \ref{espresso}).
\par
The numbers 0 through 2 in Espresso are 1 gesture long. The next 3 digits, 3 through 5, are 2 gestures long and the rest of the numbers are 3 gestures long. Assuming all digits are equally likely, this gives us an average gesture per digit of 2.1.
  \begin{align*}
    AVG(Espresso) &= \frac{1\times3 + 2\times3 + 3\times4}{10} = 2.1
  \end{align*}
  
\begin{table}[ht]
  \caption{Espresso Codes}
  \centering
  \begin{tabular}{cl}
  \hline
  \multicolumn{1}{c}{Digit} & \multicolumn{1}{c}{Code} \\
  \hline
  0 & 1-finger swipe \\
  1 & 1-finger tap \\
  2 & 2-finger tap \\
  3 & 3-finger tap + 1-finger swipe \\
  4 & 3-finger tap + 1-finger tap \\
  5 & 3-finger tap + 2-finger tap \\
  6 & 3-finger tap + 3-finger tap + 1-finger swipe \\
  7 & 3-finger tap + 3-finger tap + 1-finger tap \\
  8 & 3-finger tap + 3-finger tap + 2-finger tap \\
  9 & 3-finger tap + 3-finger tap + 3-finger tap \\ [1ex]
  \hline
  \end{tabular}
  \label{espresso}
\end{table}

\subsection{Cappuccino}
We developed an optimal prefix-free code on 4 symbols that requires lower number of gestures per digit on average called Cappuccino. In Cappuccino, one-finger swipe represents either 0 or 10. A two-gesture digit is derived by subtraction, with the one-finger swipe representing 10. For example, 9 is represented by a swipe followed by a one-finger tap $(10 - 1)$. Similarly, 8 is represented by a swipe and a two-finger tap $(10 - 2)$ and 7 is represented by a swipe and a three-finger tap. In all other cases, one-finger swipe represents 0. For instance, the digit 3 is represented by three-finger tap and a swipe $(3 + 0)$. However, the digit 0 is represent by 2 one-finger swipes. The second one-finger swipe is added to make the code prefix free. Furthermore, the number 6 is represented by two three-finger taps. Espresso and Cappuccino have the same code for the digits 1 through 5 (see table \ref{cappuccino}).
\par
The numbers 1 and 2 in Espresso are 1 gesture long. The next 3 digits, 3 through 9, are 2 gestures long. Assuming that all the digits are equally likely, this gives us an average gesture per digit of 1.8.
  \begin{align*}
    AVG(Cappuccino) &= \frac{1\times2 + 2\times8}{10} = 1.8
  \end{align*}

\begin{table}[ht]
  \caption{Cappuccino Codes}
  \centering
  \begin{tabular}{cl}
  \hline
  \multicolumn{1}{c}{Digit} & \multicolumn{1}{c}{Code} \\
  \hline
  0 & 1-finger swipe + 1-finger swipe \\
  1 & 1-finger tap \\
  2 & 2-finger tap \\
  3 & 3-finger tap + 1-finger swipe \\
  4 & 3-finger tap + 1-finger tap \\
  5 & 3-finger tap + 2-finger tap \\
  6 & 3-finger tap + 3-finger tap \\
  7 & 1-finger swipe + 3-finger tap \\
  8 & 1-finger swipe + 2-finger tap \\
  9 & 1-finger swipe + 1-finger tap \\ [1ex]
  \hline
  \end{tabular}
  \label{cappuccino}
\end{table}