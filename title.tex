% !TEX root =  paper.tex
\label{sec:title}
\begin{center}   % Center everything in the page
  {\huge DigiTaps}\\[1.5cm]
  {\large Vaspol Ruamviboonsuk}\\[0.2cm]
  {\large Computer Science and Engineering}\\[0.2cm]
  {\large University of Washington}\\[0.5cm]
  {\large \today}\\[1.0cm]
\end{center}

\begin{abstract}
Visually impaired people use touchscreens on smartphones with a screen reader. The screen reader provides audio feedback to the user for navigating the phone. However, screen readers are difficult to use in a noisy setting because the users must pay careful attention to the audio feedback. We introduce \textit{DigiTaps}, an eyes-free number entry method with minimal audio feedback. The digits are represented by a combination of simple primitive gestures such as tap and swipe. Users can perform the gestures anywhere on the screen. We conducted a preliminary lab study to compare the DigiTaps methods to the screen reader method. The participants achieved a significantly lower error rate using the DigiTaps method than using the screen reader method, while still achieving a higher entry rate. To evalutate the gestures further in a real-world setting, we develop DigiTaps game; a number entry game designed for conducting user studies in the wild. We made it available on the Apple App Store and use it as a platform to collect the players’ data such as touch events and game playing statistics. With the data collected, we analyze the data to evalutate the two DigiTaps methods.
\end{abstract}

\clearpage % no content after the abstract
