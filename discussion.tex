% !TEX root =  paper.tex
\section{Discussion}
\label{sec:discussion}

One interesting finding is that most of the players enter the numbers more accurately using the Cappuccino method, despite being less intuitive than the Espresso method. Furthermore, P1, P2 and P5 played more DigiTaps games using the Cappuccino method. Since the Cappuccino method requires 1.8 taps per digit on average, players transitioning from the Espresso method to the Cappuccino method is expected. However, P3 briefly tried the Cappuccino method, but eventually switched back to playing the Espresso method and P6 did not enter any number using the Cappuccino method. The accuracy of the two methods do not differ greatly and the Cappuccino method is marginally more accurate than the Espresso method.
\par
In terms of speed, we get a mixed result which we cannot justify that Cappuccino is evidently faster than Espresso or Espresso is faster than Cappuccino. P3 entered numbers using the Espresso method almost two times faster than using the Cappuccino method. On contrary, P5 can enter the numbers using the Cappuccino method 0.1 digits per second faster than using the Espresso method. There is not a clear that Espresso is better than Cappuccino and vice-versa. 
\par
The levels in the DigiTaps game are not played equally. Most the players played the first few levels more often than the later levels because the numbers presented in the early levels has fewer number of digits. Numbers in level 1 consists of only 3 digits whereas numbers in level 5 consists of 7 digits. As the player progresses through the levels, DigiTaps game becomes number memorization game instead of a number entry game.


\begin{comment}
Outline
  * No significant difference between Espresso and Cappuccino
  * Some players are more comfortable and reluctant to move to Cappuccino
  * Players doesn't perform as fast as we wanted due to want to correctly enter the numbers?
  
  [done]
  * Players tend to play the first couple of levels more because less digits are presented.

\end{comment}