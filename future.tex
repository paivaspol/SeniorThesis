% !TEX root =  paper.tex
\section{Conclusion and Future Work}
\label{sec:future}
We introduced DigiTaps code, a prefix-free code that uses minimal voice feedback. DigiTaps requires at most 2.1 taps per digit and uses only simple gestures in its codes such as swipe and tap. DigiTaps shows promising results in the preliminary evaluation with the entry rate of 1.99 seconds per digit ($SD = 1.25$) using the Espresso method whereas using VoiceOver achieves only 2.77 seconds per digit ($SD = 1.24$).

We then developed DigiTaps game as a platform for conducting user study in the large for the DigiTaps gestures. DigiTaps is a simple number entry game. It gives a number to the player and the player uses one of the DigiTaps gestures to enter the number. DigiTaps is distributed through the Apple App Store. We collected information on how the gestures were performed and other metadata such as the number given to the player in the game. At the time of writing this paper, there are 654 installations and 129,906 events and some preliminary evaluation has been done on the data.

Even though DigiTaps code and the DigiTaps game has been developed and studied to some extent, there are several aspects of the project that can be improved.
\begin{enumerate}
  \item \textbf{Advertise and recruit recurring players:}
  \par
   As presented in the preliminary data analysis section, the data we gathered is still too little to do any in-depth evaluation on the gestures. We need to recruit more players and keep them playing the game.

  \item \textbf{Implement DigiTaps codes in a real-world application:}
  \par
  DigiTaps code shows potential to be fast and works well in noisy settings. We can incorporate DigiTaps code to any number entering application such as a calculator, personal identification number (PIN) entry and making phone calls.

  \item \textbf{Clean up DigiTaps framework and open source it:}
  \par
  DigiTaps was designed to be a platform to do user study in the large from the beginning. Users can use implement their own game and replace the DigiTaps game engine with the implemented game engine. In addition, users can define their own gestures to be detected in DigiTaps. However, at the current state, the framework still needs some cleaning up to make the framework easy to use as much as possible.
\end{enumerate}


\begin{comment}
  Outline:
    - In depth data analysis on the data gathered from DigiTaps
    - Use DigiTaps gesture on real-world applications for entering numbers such as a calculator.
    - DigiTaps was designed to be a framework for doing text or number entry user study. Other people can take DigiTaps and use it for their user study in the large.
\end{comment}