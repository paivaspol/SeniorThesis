% !TEX root =  paper.tex
\section{Related Work}
\label{sec:related}

\subsection{Eyes-free Text Entry Methods}
% Introduce the state of the world with screen readers
Blind users use smartphones by using a screen reader that pre-installed on smartphones such as iOS's Voice Over \cite{VoiceOver:2014} and Android's TalkBack \cite{TalkBack:2014}. Both screen readers use the interaction techniques similar to what Kane et al.\ presented in \cite{Kane:2008}. To interact with the screen, a user explores the screen with his finger and gets audio feedback from the phone when an element is touched. The user has to carefully listen to the audio feedback. When the target is reached, the user performs a selection gesture which can either be a double-tap anywhere on the screen or a split-tap (tap another finger on the screen while the first finger is still on the selection target). Even though the screen readers are widely adopted in the blind community, using screen readers to enter text achieves a low text entry rate. Studies have shown that blind users can enter text at only 4.5 words-per-minute using VoiceOver \cite{Azenkot:2012}. Furthermore, using screen reader method is error-prone \cite{Oliveira:2011}.

% Present other kinds of text entry methods: Braille-based
% Includes PerkInput, BrailleTouch
Many eyes-free text entry methods have been introduced recently. Azenkot et al.\ presented PerkInput, a chorded input for touchscreens where Braille cells are input one column at a time \cite{Azenkot:2012}. Southern et al.\ presented BrailleTouch, an application for typing braille based on the six-key braille keyboard \cite{Southern:2012}. These text entry methods require finger calibrations or allow the user to touch in only certain locations on the screen. In contrast with DigiTaps, users are not required to calibrate their fingers with the phone and can perform the gesture anywhere on the screen. Furthermore, knowledge of Braille is a requirement for using PerkInput and BrailleTouch while DigiTaps does not require the users to know Braille.

% Present eyes-free text entry H-4. H-4 uses Huffman code for entering text, it is optimal, but huffman code are not learnable / not intuitive to learn.
MacKenzie et al.\ introduced an eyes-free text entry method using a joystick called H4-writer \cite{MacKenzie:2011}. This method achieved a 20WPM text entry rate. H4-writer uses an optimal prefix-free code, Huffman Code, for encoding the words. Unlike the DigiTaps code, the Huffman Code does not relate to the semantics of the symbols. Thus, it is not as intuitive as DigiTaps' code which all the codes are related to the actual meanings of the symbols.

% DigiTaps is doesn't have complex gesture or shape based gestures unlike graffiti
Graffiti and Unistroke \cite{Goldberg:1993} uses print symbols for representing each character. Unlike Graffiti and Unistroke, DigiTaps uses simple gestures like tap and swipe to represent its symbols. Blind users do not have the knowledge of how the print symbols look like. While using the print symbols in the text entry method is efficient for sighted users, blind users can have trouble performing them because their drawing skill would be less accurate than sighted users \cite{Kane:2011}. 

\subsection{Games for User Study in the Wild}
Henze et al.\ conducted a user study in the wild on analyzing touch performance \cite{Henze:2011} using a text entry Android game distributed via Google Play Store. The game collected important data from the players such as touch positions to analyze the accuracy and the performance of the touch events. We adopted the application design presented in \cite{Henze:2012} and developed DigiTaps game as a platform for conducting number entry method user studies in the wild.