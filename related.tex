% !TEX root =  paper.tex
\section{Related Works}
\label{sec:related}

\subsection{DigiTaps Code}
% Introduce the state of the world with screen readers
Blind users use smartphones by using the screen reader that is built-in with the phone such as Apple's Voice Over \cite{VoiceOver:2014}. Android also has TalkBack, which is a screen reader on Android \cite{TalkBack:2014}. Both screen readers use the interaction techniques that Kane et al. presented in Slide Rule \cite{Kane:2008}. To perform a selection, a user has to use one finger to explore the screen. The screen reader provides audio feedback of what is beneath the user's finger. The user has to carefully listen to the audio feedback. When the target is reached, the user performs a selection gesture which can either be a double-tap anywhere on the screen or a split-tap (tap another finger on the screen while the first finger is still on the selection target). Even though these screen readers are widely adopted in the blind community, using screen readers to enter text achieves a low the text entry rate. Blind users can enter text at only 4.5 words-per-minute using VoiceOver \cite{Azenkot:2012}. Furthermore, using screen reader method is onerous and error-prone \cite{Oliveira:2011}.

% Present other kinds of text entry methods: Braille-based
% Includes PerkInput, BrailleTouch
Many eyes-free text entry method has been introduced recently. Azenkot et al. presented PerkInput, a chorded input for touchscreens where Braille cells are input one column at a time \cite{Azenkot:2012}. Southern et al. presented BrailleTouch, an application for typing braille based on the six-key braille keyboard \cite{Southern:2012}. These text entry methods require finger calibrations or allow the user to touch in only certain locations on the screen. In contrast, users are not required to calibrate their fingers with the phone and can perform the gesture anywhere on the screen. Furthermore, PerkInput and BrialleTouch requires knowledge on braille to use them while DigiTaps does not require the users to know braille.

% Present eyes-free text entry H-4. H-4 uses Huffman code for entering text, it is optimal, but huffman code are not learnable / not intuitive to learn.
MacKenzie et al. introduces an eyes-free text entry method using a joystick called H4-writer \cite{MacKenzie:2011}, which achieves a 20WPM text entry rate. Like DigiTaps, H4-writer uses an optimal prefix-free code, specifically Huffman Code, as the input to input text. Unlike DigiTaps code, however, the Huffman Code does not relate to the actual notations of the symbols. Thus, it is unlikely to be as intuitive as DigiTaps' code which all the codes are related to the actual notations of the symbols.

% DigiTaps is doesn't have complex gesture or shape based gestures unlike graffiti
Blind users do not have the knowledge of how the print symbols look like. While using the print symbols in the text entry method is efficient for sighted users, blind users can have trouble performing them because their drawing skill would be less accurate than sighted users \cite{Kane:2011}. Graffiti and Unistroke \cite{Goldberg:1993} uses print symbols for representing each character. Unlike Graffiti and Unistroke, DigiTaps uses simple gestures like taps and swipes to represent its symbols.

\subsection{DigiTaps Game}
Henze et al. conducted a user study in the large on analyzing touch performance \cite{Henze:2011}. An text entry Android game was developed and distributed via Google Play Store. The game collected important data from the players such as touch positions to analyze the accuracy and the performance of the touch events. We adopted the study design presented in \cite{Henze:2012} and developed DigiTaps game as a platform for conducting number entry method user studies in the large.